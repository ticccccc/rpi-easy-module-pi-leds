%% ticccccc in github
% !TeX program = xelatex
% !TeX encoding = utf8

\documentclass{ctexart}
\usepackage{circuitikz}


\thispagestyle{empty}% 当前页不显示页码
\begin{document}

%GPIO序号对应于BCM编码, IO序号对应wiringPi编码, 数字为物理引脚, 括号内为功能名.


\begin{center}
\begin{circuitikz}[scale=1.5]
    %%%%%%%%% raspberry pi 4 circuit schematics start %%%%%%%%%
    \ctikzset{multipoles/dipchip/width=4.5}
    \ctikzset{multipoles/dipchip/pin spacing=0.6}
    % \ctikzset{bipoles/length=1.5cm}
    \draw node[dipchip, num pins=40, hide numbers, no topmark, external pins width=1, fill=cyan!20!white](rpi){\LARGE rpi4};
    \node [right, color=orange] at (rpi.bpin 1) {3.3V};
    \node [above, color=orange] at (rpi.pin 1) {3.3V};
    \node [left, color=red] at (rpi.bpin 40) {5V};
    \node [above, color=red] at (rpi.pin 40) {5V};
    \node [right, color=cyan] at (rpi.bpin 2) {GPIO.2(SDA.1)};
    \node [above, color=cyan] at (rpi.pin 2) {IO.8};
    \node [left, color=red] at (rpi.bpin 39) {5V};
    \node [above, color=red] at (rpi.pin 39) {5V};
    \node [right, color=cyan] at (rpi.bpin 3) {GPIO.3(SCL.1)};
    \node [above, color=cyan] at (rpi.pin 3) {IO.9};
    \node [left] at (rpi.bpin 38) {GND};
    \node [above] at (rpi.pin 38) {GND};
    \node [right, color=olive] at (rpi.bpin 4) {GPIO.4};
    \node [above, color=olive] at (rpi.pin 4) {IO.7};
    \node [left, color=blue] at (rpi.bpin 37) {(TXD)GPIO.14};
    \node [above, color=blue] at (rpi.pin 37) {IO.15};
    \node [right] at (rpi.bpin 5) {GND};
    \node [above] at (rpi.pin 5) {GND};
    \node [left, color=blue] at (rpi.bpin 36) {(RXD)GPIO.15};
    \node [above, color=blue] at (rpi.pin 36) {IO.16};
    \node [right, color=olive] at (rpi.bpin 6) {GPIO.17};
    \node [above, color=olive] at (rpi.pin 6) {IO.0};
    \node [left, color=olive] at (rpi.bpin 35) {GPIO.18};
    \node [above, color=olive] at (rpi.pin 35) {IO.1};
    \node [right, color=olive] at (rpi.bpin 7) {GPIO.27};
    \node [above, color=olive] at (rpi.pin 7) {IO.2};
    \node [left] at (rpi.bpin 34) {GND};
    \node [above] at (rpi.pin 34) {GND};
    \node [right, color=olive] at (rpi.bpin 8) {GPIO.22};
    \node [above, color=olive] at (rpi.pin 8) {IO.3};
    \node [left, color=olive] at (rpi.bpin 33) {GPIO.23};
    \node [above, color=olive] at (rpi.pin 33) {IO.4};
    \node [right, color=orange] at (rpi.bpin 9) {3.3V};
    \node [above, color=orange] at (rpi.pin 9) {3.3V};
    \node [left, color=olive] at (rpi.bpin 32) {GPIO.24};
    \node [above, color=olive] at (rpi.pin 32) {IO.5};
    \node [right, color=purple] at (rpi.bpin 10) {GPIO.10(MOSI)};
    \node [above, color=purple] at (rpi.pin 10) {IO.12};
    \node [left] at (rpi.bpin 31) {GND};
    \node [above] at (rpi.pin 31) {GND};
    \node [right, color=purple] at (rpi.bpin 11) {GPIO.9(MISO)};
    \node [above, color=purple] at (rpi.pin 11) {IO.13};
    \node [left, color=olive] at (rpi.bpin 30) {GPIO.25};
    \node [above, color=olive] at (rpi.pin 30) {IO.6};
    \node [right, color=purple] at (rpi.bpin 12) {GPIO.11(SCLK)};
    \node [above, color=purple] at (rpi.pin 12) {IO.14};
    \node [left, color=purple] at (rpi.bpin 29) {(CE0\_N)GPIO.8};
    \node [above, color=purple] at (rpi.pin 29) {IO.10};
    \node [right] at (rpi.bpin 13) {GND};
    \node [above] at (rpi.pin 13) {GND};
    \node [left, color=purple] at (rpi.bpin 28) {(CE1\_N)GPIO.7};
    \node [above, color=purple] at (rpi.pin 28) {IO.11};
    \node [right, color=cyan] at (rpi.bpin 14) {GPIO.0(SDA.0)};
    \node [above, color=cyan] at (rpi.pin 14) {IO.30};
    \node [left, color=cyan] at (rpi.bpin 27) {(SCL.0)GPIO.1};
    \node [above, color=cyan] at (rpi.pin 27) {IO.31};
    \node [right, color=olive] at (rpi.bpin 15) {GPIO.5};
    \node [above, color=olive] at (rpi.pin 15) {IO.21};
    \node [left] at (rpi.bpin 26) {GND};
    \node [above] at (rpi.pin 26) {GND};
    \node [right, color=olive] at (rpi.bpin 16) {GPIO.6};
    \node [above, color=olive] at (rpi.pin 16) {IO.22};
    \node [left, color=olive] at (rpi.bpin 25) {GPIO.12};
    \node [above, color=olive] at (rpi.pin 25) {IO.26};
    \node [right, color=olive] at (rpi.bpin 17) {GPIO.13};
    \node [above, color=olive] at (rpi.pin 17) {IO.23};
    \node [left] at (rpi.bpin 24) {GND};
    \node [above] at (rpi.pin 24) {GND};
    \node [right, color=olive] at (rpi.bpin 18) {GPIO.19};
    \node [above, color=olive] at (rpi.pin 18) {IO.24};
    \node [left, color=olive] at (rpi.bpin 23) {GPIO.16};
    \node [above, color=olive] at (rpi.pin 23) {IO.27};
    \node [right, color=olive] at (rpi.bpin 19) {GPIO.26};
    \node [above, color=olive] at (rpi.pin 19) {IO.25};
    \node [left, color=olive] at (rpi.bpin 22) {GPIO.20};
    \node [above, color=olive] at (rpi.pin 22) {IO.28};
    \node [right] at (rpi.bpin 20) {GND};
    \node [above] at (rpi.pin 20) {GND};
    \node [left, color=olive] at (rpi.bpin 21) {GPIO.21};
    \node [above, color=olive] at (rpi.pin 21) {IO.29};

    \draw (rpi.pin 1) -- (rpi.bpin 1) node[color=orange, above, midway] {1};
    \draw (rpi.pin 40) -- (rpi.bpin 40) node[color=red, above, midway] {2};
    \draw (rpi.pin 2) -- (rpi.bpin 2) node[color=cyan, above, midway] {3};
    \draw (rpi.pin 39) -- (rpi.bpin 39) node[color=red, above, midway] {4};
    \draw (rpi.pin 3) -- (rpi.bpin 3) node[color=cyan, above, midway] {5};
    \draw (rpi.pin 38) -- (rpi.bpin 38) node[above, midway] {6};
    \draw (rpi.pin 4) -- (rpi.bpin 4) node[color=olive, above, midway] {7};
    \draw (rpi.pin 37) -- (rpi.bpin 37) node[color=blue, above, midway] {8};
    \draw (rpi.pin 5) -- (rpi.bpin 5) node[above, midway] {9};
    \draw (rpi.pin 36) -- (rpi.bpin 36) node[color=blue, above, midway] {10};
    \draw (rpi.pin 6) -- (rpi.bpin 6) node[color=olive, above, midway] {11};
    \draw (rpi.pin 35) -- (rpi.bpin 35) node[color=olive, above, midway] {12};
    \draw (rpi.pin 7) -- (rpi.bpin 7) node[color=olive, above, midway] {13};
    \draw (rpi.pin 34) -- (rpi.bpin 34) node[above, midway] {14};
    \draw (rpi.pin 8) -- (rpi.bpin 8) node[color=olive, above, midway] {15};
    \draw (rpi.pin 33) -- (rpi.bpin 33) node[color=olive, above, midway] {16};
    \draw (rpi.pin 9) -- (rpi.bpin 9) node[color=orange, above, midway] {17};
    \draw (rpi.pin 32) -- (rpi.bpin 32) node[color=olive, above, midway] {18};
    \draw (rpi.pin 10) -- (rpi.bpin 10) node[color=purple, above, midway] {19};
    \draw (rpi.pin 31) -- (rpi.bpin 31) node[above, midway] {20};
    \draw (rpi.pin 11) -- (rpi.bpin 11) node[color=purple, above, midway] {21};
    \draw (rpi.pin 30) -- (rpi.bpin 30) node[color=olive, above, midway] {22};
    \draw (rpi.pin 12) -- (rpi.bpin 12) node[color=purple, above, midway] {23};
    \draw (rpi.pin 29) -- (rpi.bpin 29) node[color=purple, above, midway] {24};
    \draw (rpi.pin 13) -- (rpi.bpin 13) node[above, midway] {25};
    \draw (rpi.pin 28) -- (rpi.bpin 28) node[color=purple, above, midway] {26};
    \draw (rpi.pin 14) -- (rpi.bpin 14) node[color=cyan, above, midway] {27};
    \draw (rpi.pin 27) -- (rpi.bpin 27) node[color=cyan, above, midway] {28};
    \draw (rpi.pin 15) -- (rpi.bpin 15) node[color=olive, above, midway] {29};
    \draw (rpi.pin 26) -- (rpi.bpin 26) node[above, midway] {30};
    \draw (rpi.pin 16) -- (rpi.bpin 16) node[color=olive, above, midway] {31};
    \draw (rpi.pin 25) -- (rpi.bpin 25) node[color=olive, above, midway] {32};
    \draw (rpi.pin 17) -- (rpi.bpin 17) node[color=olive, above, midway] {33};
    \draw (rpi.pin 24) -- (rpi.bpin 24) node[above, midway] {34};
    \draw (rpi.pin 18) -- (rpi.bpin 18) node[color=olive, above, midway] {35};
    \draw (rpi.pin 23) -- (rpi.bpin 23) node[color=olive, above, midway] {36};
    \draw (rpi.pin 19) -- (rpi.bpin 19) node[color=olive, above, midway] {37};
    \draw (rpi.pin 22) -- (rpi.bpin 22) node[color=olive, above, midway] {38};
    \draw (rpi.pin 20) -- (rpi.bpin 20) node[color=olive, above, midway] {39};
    \draw (rpi.pin 21) -- (rpi.bpin 21) node[color=olive, above, midway] {40};

    %%%%%%%%% raspberry pi 4 circuit schematics end %%%%%%%%%
    %%%%%%%%% leds blink %%%%%%%%%
    %%%%%%%%% leds blink start %%%%%%%%%
    \draw [opacity=0] (rpi.pin 13) to ++(-3, 0) node(C){};
    \draw (rpi.pin 6) to[leD, fill=red] ++(-3, 0) node{} to (C.center) to[R, l=R] (rpi.pin 13);
    \draw (rpi.pin 7) to[leD, fill=yellow] ++(-3, 0);
    \draw (rpi.pin 8) to[leD, fill=green] ++(-3, 0);
    \draw (rpi.pin 15) to[leD, fill=blue] ++(-3, 0) to (C.center);
    %%%%%%%%% leds blink start %%%%%%%%%
\end{circuitikz}
\end{center}
\end{document}